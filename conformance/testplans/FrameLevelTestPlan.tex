\documentclass[11pt]{article}
\usepackage{geometry}                % See geometry.pdf to learn the layout options. There are lots.
\geometry{letterpaper}                   % ... or a4paper or a5paper or ... 
%\geometry{landscape}                % Activate for for rotated page geometry
\usepackage[parfill]{parskip}    % Activate to begin paragraphs with an empty line rather than an indent
\usepackage{graphicx}
\usepackage{amssymb}
\usepackage{epstopdf}
\usepackage{hyperref}

\DeclareGraphicsRule{.tif}{png}{.png}{`convert #1 `dirname #1`/`basename #1 .tif`.png}

\title{OpenLCB Test plan for (Sample) Protocol}
\author{The OpenLCB Group}
\date{}                                         % Activate to display a given date or no date

\begin{document}
\maketitle


\section{Introduction}

This note documents the procedure for testing an OpenLCB implementation against the 
\href{url}{(Standard name here) Standard}.

The tests are traceable to specific sections of the Standard.

The testing assumes that the Device Under Test (DUT) is being exercised by other
nodes on the message network, 
e.g. is responding to enquiries from other parts of the message network.

\subsection{Required Equipment}

See the separate ``Installing the OpenLCB Test Software" document for initial installation 
and set up of the test program.

If a direct CAN connection will be used,
a supported USB-CAN adapter
    \footnote{See ``Installing the OpenLCB Test Software"}
is required. 
Connect the adapter to the DUT using a single UTP cable and connect two CAN terminators.

Provide power to the DUT using its recommended method.

\section{Set Up}
The following steps need to be done once to configure the test program:.
\begin{enumerate}
\item Start the test configuration program. 
\item Select ``Set Up DUT".
\item Get the Node ID from the DUT\footnote{Where do we require this to be marked on a node?} 
\item Enter that Node ID into the program.
\item Configure the test program for the USB-CAN adapter's device address
        or the TCP hostname and port.
\item Quit the test program and reply ``Y" to "Save configuration?" when prompted.
\end{enumerate}

The following steps need to be done at the start of each testing session.
\begin{enumerate}
\item Check that the DUT is ready for operation.
\item Start the test program.
\end{enumerate}

\section{CAN Frame Level Procedure}

Select ``CAN Frame Layer testing" in the test program, 
then select each section below in turn.  Follow the prompts
for when to reset/restart the node and when to check 
outputs against the node documentation.

\subsection{Alias Acquisition}
Select the "Alias Acquisition Test" option in the test program. 
This will test the sequences used by the DUT to acquire an OpenLCB node alias on the CAN link. 
The program will wait until it sees the DUT start communicating after the next step.

Follow the prompts when asked to reset or otherwise initialize the DUT.

This section's tests cover:

\begin{enumerate}
\item Part A tests the initial sequence used at start up in the absence of a collision\footnote{SSS Standard, line 321}
\item Part B tests the regeneration of an alias when there's a collision during the CID process\footnote{SSS Standard, line 321}
\item Part C ...
	\end{enumerate}


\subsection{AME Sequences}

...





\end{document}  
